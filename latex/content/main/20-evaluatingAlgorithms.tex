\section{Evaluating algorithms}\label{sec:evaluation}
    The previous section defined and collected data from many data sources and used it to feed a purpose-built simulation. We will now be using this simulation to evaluate the relative performance of several algorithms.
    
    The algorithms in question will be grouped into three categories: Static algorithms, layered algorithms, and scoring based algorithms. The categories will be explained in more detail in each of the following subsections. %\todo{Find out if these definitions mean something in existing literature}
    
    It should be noted that all graphs plotting the performance over time of algorithms are using the same value range to make comparison easier. The X-Axis of the plots shows the number of processed simulation events. The logarithmic Y-Axis depicts the percentage of missed accesses (note that this is in relation to the cumulative number of accesses up to that X coordinate and not the total number of accesses). Since the algorithms will cover a vast range of performances, the figures will be using a logarithmic scale\footnote{For this reason, the range is going from 0,5\% to 100\%.} on the Y-Axis. The real-world data source uses a disk size of approximately \SI{800}{\giga\byte}. Since we are discarding about 22,5\% of pipelines and only approximate the size, which is expected to yield an overall lower influx, we will be using the first power of two that is smaller than $22,5\% * \SI{800}{\giga\byte} = \SI{620}{\giga\byte}$ for all initial simulations and comparisons. In section \ref{sec:overall}, we will be looking at the behaviour of the algorithms with different disk sizes.
    
    \subsection{Static algorithms}\label{sec:static}
    The static algorithm category includes those that take static inputs from the simulation (e.g. chronologically ordered list of stored pipelines) and always yield a result. Most of the algorithms will be using a simple data structure to make their decisions.
    
    \colfig{image/graphs/lineplot-static}{Performance of static algorithms}{fig:lineplot-static}
    
    \subsubsection{FIFO}
        The first static algorithm to evaluate will be using a simple queue data structure. This data structure allows elements to be added at one end and retrieved at the other \cite{data-structures}. Effectively, this means that the first item that goes in is also the first item that goes out, which can be abbreviated to \textbf{F}irst-\textbf{I}n-\textbf{F}irst-\textbf{O}ut. This approach will ensure that every pipeline will be treated the same and stored for the maximum duration that is possible with the given storage limit\footnote{Note that this is the maximum possible duration when treating all pipelines the same. By favouring larger pipelines, the storage duration for smaller pipelines would be increased.}. In theory, it is expected to be a very balanced approach. Under the assumption that every developer is triggering the same pipelines with the same failure scenario and similar debugging times (and thus access requirements), this is expected to perform very well. However, it is unclear how closely the real-world matches this assumption.
        
        The data in figure \ref{fig:lineplot-static} shows that FIFO is performing relatively well in relation to other static algorithms, as expected, with a performance between 3\% and 5\% and only slight fluctuations\footnote{These fluctuations are likely caused by differences in the input data.}.
    
    \subsubsection{LIFO}
        The next algorithm in line is similar to FIFO, but instead of using a queue, it uses a stack as the underlying data structure. With a stack, elements can be appended at one end but only be retrieved from the same end in the reverse order \cite{data-structures}. In short, this means that the last item that goes in is the first item that comes back out or \textbf{L}ast-\textbf{In}-\textbf{F}irst-\textbf{O}ut. Since this implies that the first artefact that comes in is likely going to stay stored for the whole duration, while newer ones will almost immediately be purged, it is expected that this algorithm performs comparatively poorly in this use-case.
        
        The data in figure \ref{fig:lineplot-static} confirms this. The algorithm initially performs somewhat acceptable as pipelines stored in the beginning are still being accessed but quickly approaches an 80\% miss-ratio as the disk fills up. It is the worst performer out of the three static algorithms.
        
    \subsubsection{RAND}
        The last static algorithm we will consider is not using the data provided by the simulation at all. Instead, it keeps an internal state of a pseudo-random number generator\footnote{Which is seeded with the same seed as the simulation but operates independently.}. The PRNG in use is a Rust implementation of the ChaCha block cypher with 12 rounds \cite{rand-docs}. It is used to select an element from the list of stored pipelines at random. While this algorithm is not expected to perform better than any other algorithm with domain-specific knowledge, it should serve as a practical reference value. If any other algorithm performs worse than random even with contextual information about pipelines, it may indicate that some assumptions in its design have been ill-advised.
        
        As expected, the data in figure \ref{fig:lineplot-static} confirms that the random approach performs worse than FIFO, which we previously considered sensible when taking domain knowledge into account, and better than LIFO, which we deemed ineffective.
    \subsection{Layered algorithms}\label{sec:layered}
    \colfig{image/graphs/lineplot-layered}{Performance of layered algorithms}{fig:lineplot-layered}
    The next category of algorithms is layered or optional algorithms. In comparison to static algorithms, they mostly rely on domain-specific information from the simulation. For this reason, they are characterised by the fact that sometimes this data is not available. For example, an algorithm that always chooses pipelines that have been merged (see section \ref{sec:algo-merged}) may not be able to choose if no merged pipelines are currently stored. For this reason, each of the algorithms in this section will have a fallback. If it is unable to decide due to lack of information, it will delegate to the fallback. We will be using the best performer of the previous section (FIFO) as our fallback algorithm\footnote{Future research may evaluate other fallback algorithms.}, and its performance is plotted as a dotted line in figure \ref{fig:lineplot-layered} for comparison.
    
    \subsubsection{MRU}
        The first algorithm is called \textbf{M}ost-\textbf{R}ecently-\textbf{U}sed. As the name implies, it uses the access log to determine which artefact to delete. It will prefer those which have been accessed most recently. While it seems counter-intuitive to delete something that has just recently been accessed, there is a case to make. In their research paper Chou et al. noted that when a sequence is scanned looped sequentially, MRU is the best replacement algorithm \cite{mru-reasoning}. While we do not have a repeating scan at hand, it is reasonable to assume that a developer will access a pipeline once to determine the status and the root cause of the issue and then not reaccess it. For this reason, the algorithm may yield favourable results.
        
        However, it performs worse than most other algorithms and reaches a stable 30\% miss ratio (see figure \ref{fig:lineplot-layered}). It still outperforms LIFO and some of the other algorithms we will consider.
        
        The current implementation is only considering whether or not a pipeline has been accessed once or not and then chooses the one with the most recent access. However, developers may make multiple requests to a pipeline with a few minutes in between before the root-cause is determined and the artefact is no longer required (e.g. the developer looks at the test report and analyses the test failures, then he starts looking at the test logs and finally observes the database dump).
        
        To further investigate, we will derive a modified version of the MRU algorithm. This version will only consider pipelines which have at least $n$ accesses. If our theory from before holds, the performance of the algorithm should increase. Figure \ref{fig:lineplot-mru} shows MRU variations with values for $n$ ranging from 2 to 64. At first glance, the performance does increase with increasing values for $n$.
        
        \colfig{image/graphs/lineplot-mru}{Performance of MRU variations}{fig:lineplot-mru}
        
        However, as this is a layered algorithm, it can delegate its decision to a fallback algorithm. For this reason, we also have to consider the percentage of deletion requests that have been delegated. The numbers are shown in figure \ref{fig:mru-fallback} and clearly show that increasing values for $n$ also increase the delegation ratio\footnote{Which is expected since the data availability is limited.}.
        
        \colfig{image/graphs/mru-fallback}{Fallback ratio of MRU variations}{fig:mru-fallback}
        
        If we look at the MRU with $n = 64$, it is close to FIFO with only approximately 2\% difference. However, looking at the fallback ratio reveals that MRU has handled less than 1\% (2656 out of 2669) of the requests. The remaining requests have been delegated to the FIFO fallback. Despite this small number of deletion requests handled by MRU, it is performing worse. The other values for $n$ paint a similar picture. It seems that no matter the limit, MRU always performs worse than our current best contender FIFO.
    
    \subsubsection{LRU}
        The next algorithm we will observe can be considered opposite to MRU as its name \textbf{L}east-\textbf{R}ecently-\textbf{U}sed implies. It operates very much the same, but instead of choosing the one accessed most recently, it picks the one whose most recent access is the furthest in the past.
        
        Under the assumption that every pipeline is being accessed eventually, it behaves like a more domain-specific version of FIFO — waiting until a pipeline has been accessed once and then waiting until storage space runs out to delete the "oldest" pipeline concerning accesses. If the assumption holds, it would be excepted to perform better than FIFO. However, the data in figure \ref{fig:lineplot-layered} shows that it performs worse than FIFO closer to RAND.
        
        To further investigate, we will take a look at the assumption made previously using a histogram. Figure \ref{fig:lru-histogram} shows the number of pipelines in relation to the number of accesses with a bucket width of 5. To make the plot more readable 18 pipelines have been excluded which each used up a single bucket in the range $210 < x < 1000$.
        
        \colfig{image/graphs/access-count-histogram}{Access count histogram}{fig:lru-histogram}
        
        % SELECT Pipeline.id AS pid, COUNT(AccessLog.timestamp) AS c FROM Pipeline LEFT JOIN AccessLog ON AccessLog.pipeline = Pipeline.id GROUP BY pid ORDER BY c DESC
        
        The histogram shows that a majority of pipelines are accessed less than five times. 65\% (2515 out of 3854) of pipelines are never accessed. This significantly reduces the effectiveness of the LRU algorithm as it now "competes" with FIFO. It also shows why FIFO is so efficient: As only a few pipelines are accessed, there is no significant benefit to employing domain-specific knowledge. However, it also indicates that the information whether a pipeline will be accessed at all is more critical, and an algorithm based on this might be more effective than FIFO. \label{sec:lru-relevancy-algorithm}Finding an algorithm to determine this would allow fast deletion of irrelevant pipelines and increase domain-specific algorithms' effectiveness. Further research is required to determine if it is possible to develop such an algorithm.
        
    \subsubsection{LF}
        Another class of algorithms concerns itself only with the stored artefacts themselves. It uses the size of the artefacts to make a decision. The first algorithm in this class is \textbf{L}argest-\textbf{First} and as the name implies always deletes the largest stored pipeline.
        
        \colfig{image/graphs/pipeline-size-histogram}{Pipeline size histogram}{fig:pipeline-size-histogram}
        
        To gain insight into why this algorithm might be advantageous, we will be consulting the pipeline size histogram\footnote{As opposed to the size sample boxplot from earlier, this uses the accumulated pipeline sizes which are calculated by summing up all job samples of a pipeline. Additionally, it does not include pipelines with insufficient samples for any job.} in figure \ref{fig:pipeline-size-histogram} with a bucket size of 200 Megabytes. While some pipelines are located in the mid-range, most of them form two extremes at less than 600MB and around 5GB.
        
        Using this information, the advantage of purging the largest pipeline first becomes evident. Deleting a single large pipeline has a high chance of making room for a large number of small pipelines, thus maximising the number of stored pipelines at any given time.
        
        \begin{Figure}
            \begin{center}
                \begin{tabular}{ l | r | r }
                    Algorithm & Stored (avg) & Deleted \\ \hline \hline
                    FIFO & 217 & 2669\\
                    LIFO & 224 & 2705\\
                    RAND & 216 & 2672\\ \hline
                    MRU & 234 & 2594\\
                    LRU & 234 & 2593\\
                    \rowcolor{nordakademie-blue!10}LF & 755 & 1609\\
                    SF & 105 & 2826\\
                    MERGED & 203 & 2709\\
                    STATUS & 194 & 2743\\ \hline
                    SCORE & 217 & 2669\\
                \end{tabular}
            \end{center}
            \captionof{table}{Algorithm storage behaviour overview}
            \label{tbl:algorithm-delete-store-counts}
        \end{Figure}
        
        To confirm this, we may look at the data in table \ref{tbl:algorithm-delete-store-counts}. It shows the arithmetic average of the number of stored pipelines throughout the simulation and the total number of deleted pipelines at the end for each algorithm. It becomes evident that LF is by far outperforming all others regarding the number of stored pipelines. Simultaneously, the number of deleted pipelines is at a low too.
        
        Overall, this would be expected to increase the performance of the algorithm. However, figure \ref{fig:lineplot-layered} reveals that it performs worse overall than MRU with a final access miss ratio of 55\%. This phenomenon could be explained by considering the timeline of storage events. Let us consider a scenario where we start by deleting a large pipeline of 5GB to make room and then receive one small pipeline of 10MB. If we attempted to store another large pipeline of 5GB, it would not fit anymore. In a sense, this can be considered similar to MRU where the most-recently added pipeline, in this case, constrained to large pipelines\footnote{Which are accounting for almost half the pipelines according to figure \ref{fig:pipeline-size-histogram}.}, when the disk is under pressure. Additionally, it comes with similar if not worse drawbacks than MRU as very small (but potentially old) pipelines are expected to stay in storage almost indefinitely. So while we are maximising the number of concurrently stored pipelines, this algorithm can not optimise the access hit ratio.
    
    \subsubsection{SF}
        Looking at another algorithm in the same class, \textbf{S}mallest-\textbf{F}irst behaves exactly opposite to LF. It deletes the smallest pipeline in storage. It shares the same drawbacks as it maximises the size of the stored pipelines and is expected to keep the largest of all pipelines in storage indefinitely. Contrary to LF, it minimises the number of stored artefacts (as seen in table \ref{tbl:algorithm-delete-store-counts}) and is thus expected to perform even worse. This is confirmed by the data in figure \ref{fig:lineplot-layered}, with up to 75\% missed accesses. It is the worst performer of all algorithms so far. 
        
    \subsubsection{MERGED}\label{sec:algo-merged}
        The next algorithm we will take into consideration is using domain-specific data. More specifically, the status of a merge request associated with a pipeline. It is expected that artefacts belonging to a merged feature branch are no longer needed. However, this algorithm's effectiveness will be heavily constrained as only a subset of pipelines have associated merge requests. While querying the data source, it became clear that most of the pipelines with no associated merge request have been triggered automatically. Using a two-step filtering approach for these might potentially increase the performance of this algorithm. Taking a look at the fallback ratio reveals that the merge logic has handled 33,3\% (901 out of 2708) of deletion requests, the remainder was delegated to FIFO. Despite this, the algorithm outperforms its delegate during almost the entire simulation and ties with it at the end (see figure \ref{fig:lineplot-layered}). In general, it can be considered the best algorithm so far and one of the two-layered ones beating a purely static algorithm, although not by a significant margin.
        
    \subsubsection{STATUS}
        The final layered algorithm we will consider also relies on domain-specific inputs. In this case, it is the exit status of the pipeline. The algorithm will delete artefacts of successful pipelines first, purge anything that did not fail (e.g. aborted pipelines), and finally delegates to FIFO.
        
        This algorithm's logic is based on the assumption that developers will not or at the very least are unlikely to be accessing successful pipelines. By looking at the simulation input data in table \ref{tbl:pipeline-status-access-relation}, we can confirm that the vast majority of accesses are made to failed pipelines.
        
        \begin{Figure}
            \begin{center}
                \begin{tabular}{ l | r }
                    Status & Accesses \\ \hline
                    Failed & 50830\\
                    Success & 1985\\
                    Cancelled & 57\\
                    Skipped & 0\\
                \end{tabular}
            \end{center}
            \captionof{table}{Access counts by pipeline status}
            \label{tbl:pipeline-status-access-relation}
        \end{Figure}
        % SELECT Pipeline.id AS pid, Pipeline.status AS ps, COUNT(AccessLog.id) AS ac FROM Pipeline LEFT JOIN AccessLog ON Pipeline.id = AccessLog.pipeline GROUP BY ps
        
        Thus this algorithm is expected to perform very well. Comparing its actual performance against the other algorithms in figure \ref{fig:lineplot-layered} confirms this with approximately 1-2\% gain over FIFO. However, it performs slightly worse than FIFO at the end of the simulation, but the difference is marginal, especially when averaging it out over the full simulation.

    \subsection{Scoring based algorithms}\label{sec:scoring}
    In the previous sections, we looked at and evaluated many algorithms that either operated on their own or delegated to a fallback in case of insufficient data availability. Especially the latter approach yielded promising results by using domain-specific inputs. However, in all cases, only a single algorithm was choosing on its own (even if that choice would be to delegate). The algorithms never interacted with another.
    
    This section will be looking at the potential performance gains achievable by combining different algorithms' decision-making process using a scoring system. Each scoring algorithm will assign a numeric score to a pipeline. The points from all algorithms will then be summed up, and the artefact with the highest one will be removed. As opposed to the layered approach, each input will partake in the decision-making process instead of deciding between inputs and using just one. To begin with, we have to define the scoring processes.
    
    \paragraph{Status} As the pipeline status input yielded promising results in the previous section, we will use it as one input for our scoring system. We will be assigning four different scores to each of the four possible (and relevant) states of a pipeline: Failed, successful, cancelled, or running.
    
    \paragraph{Merge} Another algorithm that proved itself in prior experiments was based on the merge status of a pipelines associated branch. For this reason, we will be using this input as well. When a pipelines branch has been merged it will be assigned a preconfigured score, otherwise zero. It should be noted that this simple approach discards the order in which pipelines have been merged. Future research might want to consider this information to develop a more sophisticated scoring approach.
    
    \paragraph{Age} Finally, we will be taking the age of the artefact into account. This can be done in two different ways. Either by defining a cutoff age after which a fixed score will be applied or feeding the age into a function (e.g. $x^2$). While the first one requires the definition of a cutoff age, it is relatively simple in design and operates within fixed bounds. The second one requires more care when designing a function to use. By outputting scores that are too large, the algorithm would overshadow others. Too small scores will result in it not influencing the results. To overcome this issue, we will combine the two approaches and use the defined cutoff age as a maximum. All values between zero and this age will be interpolated using an easing function in the range [0;1]. The resulting score will be multiplied by a fixed amount. For interpolation, we will be using a cosine based function shown in equation \ref{eq:age-easing}, which eases both in and out\footnote{This function has been chosen due to its relative simplicity — however, it is expected that this choice does not make any significant impact in this context as other factors play a stronger role.}.
    
    \begin{equation}\label{eq:age-easing}
        y=-0.5\cdot\left(\cos\left(\pi\ \cdot\ x\right)-1\right)
    \end{equation}
    
    While more inputs could be added, the number of parameters to define is already growing rather large. To keep these manageable, we will keep the scoring algorithm count at three. This is presumably the biggest drawback to this type of algorithm; the parameter space's vastness makes choosing optimal values a problematic endeavour. Since finding the best parameters warrants an entire dedicated research paper, we have picked values based on previous chapters' experience and using basic manual gradient-based optimisation with a bisecting search pattern. While this approach is far from ideal or reproducible, it still revealed compelling results. 
    
    \colfig{image/graphs/lineplot-score}{Performance of scoring algorithms}{fig:lineplot-scoring}
    
    The comparison against previous best performers in figure \ref{fig:lineplot-scoring} reveals that the scoring-based algorithm performs similarly to the other contenders, depending on the simulation progress. Overall, it is on-par with FIFO and MERGED and outperformed STATUS at the end of the simulation.
    
    Even though the parameters have not been optimised properly, this shows that there is potential in a shared approach that utilises different inputs simultaneously. For this reason, it is recommended to further investigate the parameter optimisation problem and other potential inputs to this algorithm.


    \subsection{Overall results \& Discussion}\label{sec:overall}
        In the previous sections, we looked at three different categories of algorithms and their respective performances. However, so far, we only considered the relative performance over the course of the simulation at a fixed virtual disk size. To compare all algorithms, we will be using a variable disk size and abstract the data captured during the simulation by only taking the final performance at the end into account. The simulated disk sizes will be calculated using powers of two with the size-ramp batch mode of the simulation. All the resulting data points are plotted in a heatmap shown in figure \ref{fig:size-heatmap}. The X-Axis is sorted by performances at 16GB, and the colour range depicts the access miss ratio with 100\%, indicating that no access could be fulfilled during the simulation.
        
        \colfig{image/graphs/size-ramp-heatmap}{Algorithm performances by disk size}{fig:size-heatmap}
        
        As previously observed in section \ref{sec:static} and \ref{sec:layered}, LIFO, SF, and LF perform particularly bad at the previously observed \SI{512}{\giga\byte} disk size. However, this also holds for all other disk sizes where these three algorithms exhibit the worst performance with access miss ratios of over 50\% across the chart. An additional remark is the behaviour of LIFO at disk sizes smaller than \SI{128}{\giga\byte}. It can be described best as a similar phenomenon to Bélady's Anomaly, which is commonly known from CPU memory paging. The anomaly describes a situation where an increase in available resources, in this case, disk space, decreases an algorithm's performance even though it would be expected to perform better \cite{belady-anomaly}. When increasing the available disk space from \SI{32}{\giga\byte} to either \SI{64}{\giga\byte} or \SI{128}{\giga\byte}, the algorithm's performance drops by up to two per cent.
        
        Next up is our static control algorithm RAND. It outperformed the previously mentioned algorithms and especially at larger disk sizes managed to undercut MRU as well. LRU stays ahead at higher disk pressures but falls behind in relative performance as disk space increases. This indicates that, at least with the current implementation, it does not make sense to employ any of the aforementioned algorithms as a purely random approach requires fewer inputs, operates faster\footnote{As it requires less inputs, fewer queries to external data sources like databases or file systems are required thus the decision time is expected to be shorter.}, and yields better results. Requiring fewer inputs also comes with the advantage that it allows for a more straightforward decision-making pipeline, simplifying setup procedures in a new environment.
        
        At the far right, a group of similarly performing algorithms developed. FIFO, STATUS, MERGED and SCORE are all performing exceptionally well with less than 5\% of requests failing at 512GB. Of the four algorithms, STATUS performs the best with MERGED and FIFO tying and SCORE being slightly behind. It is surprising that one of the arguably simplest algorithms, FIFO, managed to tie with more complex, domain-driven algorithms. Depending on the application scenario, it might be favourable to use it over one of the other ones, as it does not require any external inputs and can treat the stored artefacts as a black box. However, even direct inputs like a pipeline's status improved its performance noticeably across the different disk sizes. As artefacts are usually pushed into the storage system from within the CI pipeline, the status is known up-front, and thus it might very well be worth the effort.\\ % \todo{Argument with a few more numbers like "1/x the disk size possible with same performance"}
        \\\label{sec:machine-learning}
        Another noteworthy observation is the potential for automated parameter discovery. In section \ref{sec:lru-relevancy-algorithm} it was discovered that a significant number of pipelines are not accessed at all and that perhaps an algorithm to decide whether or not a pipeline is relevant in general might be useful. While this is strictly speaking a subset of the problem we are trying to solve in general, the abstraction might allow for a more precise algorithm. Finding an answer to this problem would effectively split the problem into two. The first half of the decision-making process would decide whether or not the data is relevant, and the second half decides on how long to store it. When considering this, it might be possible to convert the problem into a machine learning classification challenge. Initial tests with data taken from the simulation database yielded over 78\% in precision and recall for a model trained with a black box machine learning tool using a boosted tree algorithm. This has been purely experimental without any considerations taken regarding to over-fitting or other potential pitfalls in machine learning. It serves as a proof of concept showing that this approach bears potential. While the author has neither the expertise nor space required to elaborate and evaluate in full, it is an intriguing entry point for future research.
        
        Speaking of machine learning: During our evaluation of the scoring based approach in section \ref{sec:scoring}, we encountered issues with the parameter space growing too large to choose well-optimised variables manually effectively. However, even with these unoptimised variables, we were able to perform similarly to one of our best algorithms so far shows that there is potential for further optimisation. As this is effectively a parameter optimisation problem, a machine learning algorithm might suit the task. Again, it is out-of-scope for this research paper but provides a meaningful research opportunity.
